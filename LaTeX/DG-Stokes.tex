\documentclass[11pt,a4paper,oneside,onecolumn]{scrartcl} 
\usepackage[left=20mm,right=20mm,top=20mm,bottom=25mm]{geometry} 

%------------------------------------------------------------------------------------------------
% Additional packages
%------------------------------------------------------------------------------------------------

\usepackage{graphicx}
\usepackage[colorlinks=true,bookmarks,bookmarksopen,bookmarksdepth=2]{hyperref}
\usepackage[hyperref]{xcolor}
\usepackage[T1]{fontenc} 							
\usepackage[english]{babel} 					
\usepackage{amssymb,amsmath}%,amsbsy}%amsthm		
\usepackage{empheq}								
\usepackage{bm} %for \boldsymbol{}
\usepackage{stmaryrd} %for DG jump brackets							
\usepackage{thmbox}
\usepackage{shadethm}		
\usepackage{scalerel} %for \fakebold
\usepackage{nicefrac}								
\usepackage{siunitx}  		
\usepackage{booktabs} %for nice rules in tables
\usepackage{subfigure}		

%noch auskommentieren:
\newcommand\red[1]{\textcolor{red}{#1}}
\usepackage[normalem]{ulem}
\newcommand\redsout{\bgroup\markoverwith{\textcolor{red}{\rule[0.5ex]{2pt}{0.5pt}}}\ULon}
\usepackage[colorinlistoftodos,prependcaption]{todonotes}


%------------------------------------------------------------------------------------------------
% Additional definitions
%------------------------------------------------------------------------------------------------
%------------------------------------------------------------------------------------------------
% please place your own definitions here 

\setlength\parindent{0pt}

\DeclareMathAlphabet{\mathcal}{OMS}{cmsy}{m}{n} % for neat \mathcal{} characters

% for \fakebold:
\newlength\bshft
	\bshft=.22pt\relax
	\def\fakebold#1{\ThisStyle{\ooalign{$\SavedStyle#1$\cr%
  	\kern-\bshft$\SavedStyle#1$\cr%
  	\kern\bshft$\SavedStyle#1$}}}

\newcommand{\R}{\mathbb{R}}

\newcommand\Pk[2]{{ \mathbb{P}_{#1}{#2} }}
\newcommand\PPk[2]{{ \fakebold{\mathbb{P}}_{#1}{#2} }}
\newcommand\Pdk[2]{{ \mathbb{P}_{#1}^\mathrm{dc}{#2} }}
\newcommand\RTk[2]{{ \fakebold{\mathbb{RT}}_{#1}{#2} }}
\newcommand\BDMk[2]{{ \fakebold{\mathbb{BDM}}_{#1}{#2} }}
\newcommand\FFk[2]{{ \fakebold{\mathbb{F}}_{#1}{#2} }}

\newcommand{\inv}{{ \mathrm{inv} }}
\newcommand{\GD}{{ \mathrm{GD} }}
\newcommand{\DG}{{ \mathrm{DG} }}
\newcommand{\CIP}{{ \mathrm{CIP} }}
\newcommand{\PS}{{ \mathrm{PS} }}
\newcommand{\PF}{{ \mathrm{PF} }}
\newcommand{\dvg}{{ \mathrm{div} }}
\newcommand{\maxrm}{{ \mathrm{max} }}
\newcommand{\tr}{{ \mathrm{tr} }}
\newcommand{\upw}{{ \mathrm{upw} }}
\newcommand{\cnv}{{ \mathrm{cnv} }}
\newcommand{\dif}{{\mathrm{dif}}}
\newcommand{\OMEGA}{{ \rb{\Omega} }}
\newcommand{\tbar}{{ \overline{t} }}

\newcommand\Rey{\mbox{\textit{Re}}}  

\DeclareMathOperator{\ip}{{\boldsymbol{\cdot}}}
\DeclareMathOperator{\Fip}{{\boldsymbol{:}}}
\DeclareMathOperator{\DIV}{\nabla\,\boldsymbol{\cdot}}
\DeclareMathOperator{\DIVh}{\nabla_\textit{h}\,\boldsymbol{\cdot}}

\newcommand{\ff}{{ \boldsymbol{f} }}
\newcommand{\uu}{{ \boldsymbol{u} }}
\newcommand{\vv}{{ \boldsymbol{v} }}
\newcommand{\ww}{{ \boldsymbol{w} }}
\newcommand{\xx}{{ \boldsymbol{x} }}
\newcommand{\bb}{{ \boldsymbol{\beta} }}
\newcommand{\nn}{{ \boldsymbol{n} }}
\newcommand{\zero}{{ \boldsymbol{0} }}
\newcommand{\tend}{{ T }}
\newcommand{\gbld}{{ \boldsymbol{g} }}
\newcommand{\TAU}{{ \boldsymbol{\tau} }}

\newcommand{\uETA}{{ \boldsymbol{\eta^\uu} }}
\newcommand{\uERR}{{ \boldsymbol{e}_h^\uu }}
\newcommand{\pETA}{{ \eta^p }}
\newcommand{\pERR}{{ e_h^p }}

\newcommand{\drm}{{ \mathrm{d} }}
\newcommand{\Drm}{{ \mathrm{D} }}
\newcommand{\dx}{{ \,\drm\xx }}
\newcommand{\ds}{{\,\drm\boldsymbol{s}}}
\newcommand{\dtau}{{ \,\drm\tau }}
\newcommand{\half}{{ \nicefrac{1}{2} }}
\newcommand{\eps}{{ \varepsilon }}

\newcommand{\PIs}{{ \boldsymbol{\pi}_s }}
\newcommand{\PIh}{{ \boldsymbol{\pi}_h }}
\newcommand{\piz}{{ \pi_0 }}
\newcommand{\Jh}{{ \boldsymbol{j}_h }}

\newcommand{\fT}{{ \mathfrak{T} }}

\newcommand\Cc[1]{{ C{\rb{#1}} }}

\newcommand{\lavg}{{ \big\{\hspace{-0.99ex}\big\{ }}						
\newcommand{\ravg}{{ \big\}\hspace{-0.99ex}\big\} }}		
\newcommand{\ljmp}{ \left\llbracket }	% no double {{ }}									
\newcommand{\rjmp}{ \right\rrbracket }	% no double {{ }}								
\newcommand\jmp[1]{{ \ljmp#1\rjmp }}										
\newcommand\avg[1]{{ \lavg#1\ravg }}

\newcommand\Ltwo{{ L^{2} }}	
\newcommand\LTWO{{ \boldsymbol{L}^{2} }}	
\newcommand\Lp[2]{{ L^{#1}{#2} }} 
\newcommand\LP[2]{{ \boldsymbol{L}^{#1}{#2} }} 
\newcommand\Lpz[2]{{ L_0^{#1}{#2} }}

\newcommand\Wmp[3]{{ W^{#1,#2}{#3} }}
\newcommand\WMP[3]{{ \boldsymbol{W}^{#1,#2}{#3} }}

\newcommand\Hm[2]{{ H^{#1}{#2} }}
\newcommand\Hmz[2]{{ H_0^{#1}{#2} }}
\newcommand\HM[2]{{ \boldsymbol{H}^{#1}{#2} }}
\newcommand\HMZ[2]{{ \boldsymbol{H}_0^{#1}{#2} }}

\newcommand\XX{{ \boldsymbol{X} }}
\newcommand\HDIV{{ \boldsymbol{H}{\rb{\dvg}} }}
\newcommand{\Hdiv}{{ \boldsymbol{H}{\rb{\dvg;\Omega}} }}		
			
\newcommand{\VV}{{ \boldsymbol{V} }}	
\newcommand{\WW}{{ \boldsymbol{W} }}	
\newcommand{\RR}{{ \boldsymbol{R} }}	
\newcommand{\HH}{{ \boldsymbol{H} }}								
\newcommand{\Q}{{ Q }}
\newcommand\VVkloc[2]{{ \boldsymbol{V}_{#1}{#2} }}

\newcommand{\T}{{ \mathcal{T}_h }} 
\newcommand{\F}{{ \mathcal{F}_h }}	
\newcommand{\FK}{{ \mathcal{F}_K }}	
\newcommand{\Fi}{{ \mathcal{F}_h^i }}								
\newcommand{\Fb}{{ \mathcal{F}_h^\partial }}		

\newcommand\rb[1]{{ \left(#1\right) }}
\newcommand\sqb[1]{{ \left[ #1 \right] }}
\newcommand\rsb[1]{{ \left(#1\right] }}
\newcommand\set[1]{{ \left\{ #1 \right\} }}
\newcommand\bra[1]{{ \langle #1 \rangle }}
\newcommand\abs[1]{{ \left\lvert#1\right\rvert }}
\newcommand\norm[1]{ \left\lVert#1\right\rVert }
\newcommand\enorm[1]{ \tripnorm{#1}_{e} }
\newcommand\esnorm[1]{ \tripnorm{#1}_{e,\sharp} }
\newcommand\nf[2]{{ \nicefrac{#1}{#2} }}

\newcommand{\tripnorm}[1]{{\left\vert\kern-\nulldelimiterspace\left\vert\kern-\nulldelimiterspace\left\vert #1
	\right\vert\kern-\nulldelimiterspace\right\vert\kern-\nulldelimiterspace\right\vert}}
	
\newcommand{\otoprule}{\midrule[\heavyrulewidth]}

\newcommand\restr[2]{{												
	\left.\kern-\nulldelimiterspace									
	#1
	\vphantom{\big|}
	\right|_{#2}
	}}
	
\newcommand{\goodgap}{%
	\hspace{0.01\subfigtopskip}
	\hspace{0.01\subfigbottomskip}
	}					
	

\begin{document}

%------------------------------------------------------------------------------------------------
%------------------------------------------------------------------------------------------------
\section{Stokes problem and test cases}	
%------------------------------------------------------------------------------------------------
%------------------------------------------------------------------------------------------------
\begin{empheq}[left=\empheqlbrace]{alignat*=2} 
		-\nu\Delta \uu + \nabla p &= \ff 	  \qquad 	&& \text{in}~\Omega,\\
		\DIV\uu &= 0 									&& \text{in}~\Omega,\\
		\uu &= \gbld_\Drm								&& \text{on}~\partial\Omega.
\end{empheq} 

\begin{empheq}[left=\empheqlbrace]{align*} 
	&?\,\rb{\uu,p}\in\VV\times\Q
	\text{ s.t., }\forall\,\rb{\vv,q}\in\VV\times\Q,\\
	&\nu a\rb{\uu,\vv} + b\rb{\vv,p} - b\rb{\uu,q}
		=\rb{\ff,\vv}.		
\end{empheq} 

Bilinear forms:
\begin{align*}
	a\rb{\ww,\vv} = \int_\Omega \nabla\ww\Fip\nabla\vv\dx,
	\quad
	b\rb{\ww,q}=-\int_\Omega q\rb{\DIV\ww}\dx
\end{align*}	

%------------------------------------------------------------------------------------------------
\subsection{2D test case}	
%------------------------------------------------------------------------------------------------

Choose $\Omega=\rb{0,1}^2$ and the exact solution to be 
\begin{align*}
	\uu = \begin{pmatrix}
		\pi \sin^2(\pi x) \sin(2 \pi y) \\
		-\pi \sin(2 \pi x) \sin^2(\pi y) 
	\end{pmatrix},
	\quad
	p = \cos(\pi x) \sin(\pi y).
\end{align*}

The corresponding data is 
\begin{align*}
	\ff = \begin{pmatrix}
		-\nu 2\pi^3 (2\cos(2\pi x)-1)\sin(2\pi y) -\pi\sin(\pi x)\sin(\pi y) \\
		\nu 2\pi^3\sin(2\pi x)(2\cos(2\pi y)-1) +\pi\cos(\pi x)\cos(\pi y)
	\end{pmatrix},
	\quad
	\gbld_\Drm = \zero. 
\end{align*}

%------------------------------------------------------------------------------------------------
\subsection{3D test case}	
%------------------------------------------------------------------------------------------------

Choose $\Omega=\rb{0,1}^3$ and the exact solution to be 
\begin{align*}
	\uu = \begin{pmatrix}
		\pi \sin^2(\pi x) \sin(2 \pi y) \sin(2\pi z) \\
		-\pi \sin(2 \pi x) \sin^2(\pi y) \sin(2\pi z) \\
		\sin(2 \pi x) \sin(2 \pi y) \sin^2(\pi z)
	\end{pmatrix},
	\quad
	p = \cos(\pi x) \sin(\pi y) \cos(\pi z).
\end{align*}

The corresponding data is 
\begin{align*}
	\ff = \begin{pmatrix}
		-\nu 2\pi^3 (3\cos(2\pi x)-2)\sin(2\pi y)\sin(2\pi z) -\pi\sin(\pi x)\sin(\pi y) \cos(\pi z) \\
		\nu 2\pi^3\sin(2\pi x)(3\cos(2\pi y)-2)\sin(2\pi z) +\pi\cos(\pi x)\cos(\pi y) \cos(\pi z) \\
		\nu 2\pi^2\sin(2\pi y)\sin(2\pi y)(3\cos(2\pi z)-2) -\pi\cos(\pi x)\sin(\pi y)\sin(\pi z)
	\end{pmatrix},
	\quad
	\gbld_\Drm = \zero. 
\end{align*}


\newpage
%------------------------------------------------------------------------------------------------
%------------------------------------------------------------------------------------------------
\section{Full DG method}	
%------------------------------------------------------------------------------------------------
%------------------------------------------------------------------------------------------------

Note that $\F=\Fi\cup\Fb$, where $\Fi$ are internal and $\Fb$ are boundary facets. \\

Imposing Dirichlet BCs strongly does not make sense here. Both normal and tangential Dirichlet BCs are imposed weakly.

\begin{empheq}[left=\empheqlbrace]{align*} 
	?\,\rb{\uu_h,p_h}\in\VV_h\times\Q_h
	\text{ s.t., }&\forall\,\rb{\vv_h,q_h}\in\VV_h\times\Q_h,\\
	\nu a_h\rb{\uu_h,\vv_h} + b_h\rb{\vv_h,p_h}  
		&= \rb{\ff,\vv_h} + \nu a_h^\partial\rb{\gbld_\Drm;\vv_h}, \\
	- b_h\rb{\uu_h,q_h} 
		&= -b_h^\partial\rb{\gbld_\Drm;q_h}	.	
\end{empheq} 

SIP method:
\begin{align*}
	a_h\rb{\ww_h,\vv_h} 
		&=	\int_\Omega \nabla_h \ww_h \Fip \nabla_h \vv_h \dx
			+\sum_{F\in\F} \frac{\sigma}{h_F} \oint_F \jmp{\ww_h} \ip \jmp{\vv_h} \ds\\
		 &\quad -\sum_{F\in\F} \oint_F \avg{\nabla \ww_h} \nn_F \ip \jmp{\vv_h} \ds
			-\sum_{F\in\F} \oint_F \jmp{\ww_h} \ip \avg{\nabla \vv_h} \nn_F \ds \\
	a_h^\partial\rb{\gbld_\Drm;\vv_h}
		&= \sum_{F\in\Fb} \frac{\sigma}{h_F} \oint_F \gbld_\Drm \ip \vv_h \ds	
			-\sum_{F\in\Fb} \oint_F \gbld_\Drm \ip \nabla \vv_h \nn \ds		
\end{align*} 

Pressure-velocity coupling:
\begin{align*}
	b_h\rb{\ww_h,q_h} 
		&= -\int_\Omega q_h\rb{\DIVh \ww_h} \dx
		+\sum_{F\in\F}\oint_F \avg{q_h}\rb{\jmp{\ww_h}\ip\nn_F} \ds \\
	b_h^\partial\rb{\gbld_\Drm;q_h}
		&=	\sum_{F\in\Fb}\oint_F q_h\rb{\gbld_\Drm\ip\nn} \ds
\end{align*}

%------------------------------------------------------------------------------------------------
%------------------------------------------------------------------------------------------------
\section{H(div)-DG method}	
%------------------------------------------------------------------------------------------------
%------------------------------------------------------------------------------------------------

Note that $\F=\Fi\cup\Fb$, where $\Fi$ are internal and $\Fb$ are boundary facets. \\

For $\HDIV$ methods, the normal component of Dirichlet BCs is enforced strongly whereas the tangential component of Dirichlet BCs is imposed weakly.

\begin{empheq}[left=\empheqlbrace]{align*} 
	?\,\rb{\uu_h,p_h}\in\VV_h\times\Q_h
	\text{ s.t., }&\forall\,\rb{\vv_h,q_h}\in\VV_h\times\Q_h,\\
	\nu a_h\rb{\uu_h,\vv_h} + b\rb{\vv_h,p_h}  
		&= \rb{\ff,\vv_h} + \nu a_h^\partial\rb{\gbld_\Drm;\vv_h}, \\
	- b\rb{\uu_h,q_h} 
		&= 0.	
\end{empheq} 

\end{document}